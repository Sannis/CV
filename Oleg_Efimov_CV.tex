\documentclass[11pt,a4paper]{moderncv}

% see https://gb.mirrors.cicku.me/ctan/macros/latex/contrib/moderncv/manual/moderncv_userguide.pdf
\moderncvstyle{casual}
\moderncvcolor{grey}

% character encoding
\usepackage{cmap}
\usepackage[utf8]{inputenc}
\usepackage[T2A]{fontenc}
\usepackage[english,russian]{babel}

% nice cites
\usepackage{cite}

% adjust the page margins
\usepackage[scale=0.9]{geometry}
% if you want to change the width of the column with the dates
\setlength{\hintscolumnwidth}{7em}
\recomputelengths

% START{personal data}
\name{Олег}{Ефимов}
%\title{Resumé title (optional)}
%\born{25/10/1988}
\phone[mobile]{+7-(910)-449-75-89}
\email{efimovov@gmail.com}
\address{London}{UK}
\social[linkedin]{oefimov}
%\photo[130pt]{photo}
% END{personal data}

% to show numerical labels in the bibliography; only useful if you make citations in your resume
\makeatletter
\renewcommand*{\bibliographyitemlabel}{\@biblabel{\arabic{enumiv}}}
\makeatother

% bibliography with mutiple entries
%\usepackage{multibib}
%\newcites{book,misc}{{Books},{Others}}

%\nopagenumbers{}


%----------------------------------------------------------------------------------
%            content
%----------------------------------------------------------------------------------
\begin{document}
\maketitle

\hypersetup{
    pdftitle={Олег Ефимов "--- curriculum vit\ae{}}
}

\section{Образование}
\cventry{2005--2011}{МГУ им.~М.В.~Ломоносова, физический факультет}{Москва}{специалист}{средний бал 4.5}{Кафедра общей физики и волновых процессов \cite{MSUOfvp}.}


\section{Опыт работы}

\cventry{03/2011--03/2012}{PHP разработчик / администратор}{PHP/MySQL/Javascrip}{Wikimapia.org \cite{WikimapiaOrg}}{}{Anykey программирование на PHP.\newline{}%
  Основные занятия:%
  \begin{itemize}%
    \item Разработка различных компонентов для сайта
    \item Администрирование тестового сервера
    \item Мониторинг работы продакшн-серверов, оперативное реагирование и консультации по решению проблем
  \end{itemize}
  Отдельно стоит выделить систему ситигидов на основе данных сайта \cite{WikimapiaOrgCityguides}.
}

\cventry{02/2007--10/2010}{Веб-разработчик}{PHP/JavaScript/AJAX}{фриланс}{}{Занимался разработкой модификаций и компонентов для форумов IPB.\newline{}%
  Основные занятия:%
  \begin{itemize}%
    \item Разработка компонентов для Invision Power Board (IP.Board)
    \item Установка модификаций любой сложности
    \item Консультирование по вопросам модифицирования и оптимизации работы форума
  \end{itemize}
  Отдельно стоит выделить первую в мире модификацию ответа с помощью AJAX для~IPB 2.1--2.3, впоследствии реализованную и для IPB 1.3.
  Полный список модификаций можно найти \newline{} на сайтах \cite{IPBSannisRu, IPBSannisIBR} или просмотреть часть из них онлайн на Github \cite{IPBSannisGithub}.
}

\cventry{10/2007--05/2008}{Программист}{МГУ}{Москва}{}{Сотрудничал с лабораторией адаптивной оптики \cite{MSUAdaptiveOptics, MSUMedPhys} в лице Н.Г.~Ирошникова.\newline{}%
  Результаты работы:%
  \begin{itemize}%
    \item Написан набор библиотек для обработки изображений:
      \begin{itemize}%
        \item поиск опорных точек по максимуму энтропии Шеннона в скользящем окне
        \item совмещение нескольких изображений в одно по известным опорным точкам
      \end{itemize}
    \item Написано GUI-приложение для этих библиотек, позволяющий восстанавливать цветное изображение по раздельно снятым цветным каналам
  \end{itemize}
}

\cventry{02/2009--05/2009}{Преподаватель компьютерного практикума}{МГУ}{Москва}{}{Работал на кафедре общей физики и волновых процессов \cite{MSUOfvp}.\newline{}%
  Обязанности:%
  \begin{itemize}%
    \item Проведения практикума по параллельному програмимрованию для студентов 3 курса
    \item Ознакомление студентов с работой в ОС GNU/Linux
    \item Обучение применения технологий OpenMP/MPI для написания параллельных приложений
  \end{itemize}
  С программой курса можно ознакомиться на сайте \cite{MSUKtg1Prac}.
}

\cventry{03/2010--$\infty$}{Разработчик}{C++/JavaScript/Google V8}{оpen-source разработки}{}{Разработка приложений и модулей для Node.js \cite{NodeJS}.\newline{}%
  Проекты:%
  \begin{itemize}%
    \item Разработка асинхронного модуля для обращения к MySQL \cite{GithubNodeMySQL}
    \item Различный вклад в проекты Nodeunit \cite{GithubNodeunit}, Nodelint \cite{GithubNodelint} и Dox \cite{GithubSannisDox}
    \item Участие в переводе на русский язык документации по Node.js \cite{GithubNodejsDocsRu}
  \end{itemize}
}


\section{Языки}
\cvlanguage{Русский}{Родной}{}
\cvlanguage{Английский}{Технический}{Относительно свободное чтение технической литературы в~области физики, математикик и компьютерной техники.
                                      При общении c авторами проектов, в~которых я участвовал на Github, также проблем с пониманием не возникало.}


\section{Компьютерные навыки}
\cvline{ОС, администрирование}{Постоянно работаю в GNU/Linux (openSUSE 11.x) с 2009 г.,
                               есть опыт развёртывания, настройки и обслуживания вычислительного кластера на основе Rocks (CentOS 5.x).
                               Имею опыт написания Makefile-ов и спецификаций для RPM-пакетов.
                               Имею опыт настройки Apache, Nginx, MySQL и Bind.
                               Администрировал кластеры из 8-20 рмашин с различными ролями.}
\cvline{Языки программирования}{Пишу на C с 2004 г., на PHP с 2007 г. Хорошо знаю внутреннее устройство Node.js.
                                Имею представление и небольшой опыт работы с C++ и JavaScript. Эпизодически использовал Delphi и C\#. Знаю SQL, включая оптимизацию запросов.}
\cvline{Web-сайты}{Долгое время занимался разработкой компонентов и модификаций для форумов IP.Board версий 1.3 и 2.1--2.3. Написал около 30 модификаций и несколько больших компонентов.
                    Есть опыт использования JavaScript для AJAX и написания бекенда к нему.
                    Валидная вёрстка с использованием HTML/XHTML не представляет сложностей, однако вопросы кроссбраузерности обычно черпаю из сети.}
\cvline{Остальное}{Использую Git в повседневной работе, не пугаюсь merge и rebase. Имею представление о~unit-тестировании и опыт написания тестов к своим библиотекам. Работал с SVN, Redmine, Nagios, Munin.}


\section{Диплом специалиста}
\cvline{название}{Самофокусировка лазерных импульсов с регулярной поперечной структурой\newline{}
                   и~сравнительный анализ филаментации на длинах волн 0.8~и~10~мкм в воздухе.}
\cvline{руководитель}{к.\,ф.-м.\,н., доцент С.А.~Шлёнов.}
\cvline{квалификация}{Физик по специальности <<физика>>.}
\cvline{}{\small Дипломная работа состояла в численном исследовании процесса самофокусировки лазеного излучения в воздухе,
                   с применением неравномерных разностных сеток и распределённых вычислений для уменьшения времени расчётов.
                   Материалы дипломной работы доступены на Github \cite{DiplomaOnGithub}.}


\section{Хобби}
\cvline{Спортивный туризм}{\small Занимаюсь пешим, горным и немного водным и вело туризмом. 2ПР, 2ВУ. А также городское ориентирование.}
\cvline{Фотография}{\small \url{http://500px.com/Sannis}, \url{http://sannis.livejournal.com}.}
\cvline{}{\small }


\renewcommand\refname{Ссылки}
\bibliographystyle{plain}
\begin{thebibliography}{99}
  \bibitem{MSUOfvp}
    Кафедра общей физики и волновых процессов физического факультета МГУ.\\
    \url{http://ofvp.phys.msu.ru}.
  \bibitem{DiplomaOnGithub}
    Дипломная работа.\\
    \url{https://github.com/Sannis/msu_phys_ofvp_diploma}.
    2010.
  \bibitem{IPBSannisRu}
    Список наболее популярных модификаций для IPB.\\
    \url{http://ipb.sannis.ru}.
  \bibitem{IPBSannisIBR}
    Список модификаций для IPB на сайте АйБиРесурс.\\
    \url{http://forums.ibresource.ru/index.php?app=core&module=search&do=user_activity&search_app=downloads&mid=36662}.
  \bibitem{IPBSannisGithub}
    Репозиторий с частью open-source модификаций для IPB.\\
    \url{https://github.com/Sannis/ipb_modifications}.
  \bibitem{MSUAdaptiveOptics}
    Лаборатория адаптивной оптики кафедры ОФиВП физического факультета МГУ.\\
    \url{http://optics.ru}.
  \bibitem{MSUMedPhys}
    Кафедра медицинской физики физического факультета МГУ.\\
    \url{http://medphys.phys.msu.ru}.
  \bibitem{MSUKtg1Prac}
    Практикум <<Параллельное программирование>>.\\
    \url{http://ktg1.phys.msu.ru/?edu_3ofvp}.
  \bibitem{NodeJS}
    Сайт Node.js.\\
    \url{http://nodejs.org}.
  \bibitem{GithubNodeMySQL}
    Асинхронный модуль для обращения к MySQL из Node.js.\\
    \url{https://github.com/Sannis/node-mysql-libmysqlclient}.
  \bibitem{GithubNodeunit}
    Утилита для unit-тестирования Node.js приложений.\\
    \url{https://github.com/caolan/nodeunit}.
  \bibitem{GithubNodelint}
    Утилита для проверки соответсвия кода правилам JSLint.\\
    \url{https://github.com/tav/nodelint}.
  \bibitem{GithubSannisDox}
    Генератор документации на основе JSDoc.\\
    \url{https://github.com/Sannis/Dox}.
  \bibitem{GithubNodejsDocsRu}
    Русская версия документации по Node.js.\\
    \url{https://github.com/kurokikaze/nodejs-docs-rus}.
  \bibitem{WikimapiaOrg}
    Сайт Wikimapia.org.\\
    \url{http://wikimapia.org}.
  \bibitem{WikimapiaOrgCityguides}
    Ситигид Москвы.\\
    \url{http://moscow.wikimapia.org}.
\end{thebibliography}

\end{document}


%% end of file `template_en.tex'.
