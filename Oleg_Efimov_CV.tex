\documentclass[11pt,a4paper]{moderncv}

%=======
% Styles
%=======

% see https://gb.mirrors.cicku.me/ctan/macros/latex/contrib/moderncv/manual/moderncv_userguide.pdf
\moderncvstyle{classic}
\moderncvcolor{grey}

% character encoding
\usepackage{cmap}
\usepackage[utf8]{inputenc}
\usepackage[T2A]{fontenc}
\usepackage[english,russian]{babel}

% adjust the page margins
\usepackage[scale=0.8]{geometry}

% if you want to change the width of the column with the dates
\setlength{\hintscolumnwidth}{7em}
\recomputelengths

%==============
% Personal data
%==============
\name{Oleg}{Efimov}
%\title{Resumé title (optional)}
\phone[mobile]{+44-7435-348-567}
\email{efimovov@gmail.com}
\address{London, UK}
\social[linkedin]{oefimov}
\social[github]{Sannis}
% END{personal data}

\nopagenumbers{}

%========
% Content
%========

\begin{document}

\maketitle

\section{Опыт работы}

\cventry{03/2011--03/2012}{PHP разработчик / администратор}{PHP/MySQL/Javascrip}{Wikimapia.org}{}{Anykey программирование на PHP.\newline{}%
  Основные занятия:%
  \begin{itemize}%
    \item Разработка различных компонентов для сайта
    \item Администрирование тестового сервера
    \item Мониторинг работы продакшн-серверов, оперативное реагирование и консультации по решению проблем
  \end{itemize}
  Отдельно стоит выделить систему ситигидов на основе данных сайта.
}

\cventry{10/2007--05/2008}{Программист}{МГУ}{Москва}{}{Сотрудничал с лабораторией адаптивной оптики в лице Н.Г.~Ирошникова.\newline{}%
  Результаты работы:%
  \begin{itemize}%
    \item Написан набор библиотек для обработки изображений:
      \begin{itemize}%
        \item поиск опорных точек по максимуму энтропии Шеннона в скользящем окне
        \item совмещение нескольких изображений в одно по известным опорным точкам
      \end{itemize}
    \item Написано GUI-приложение для этих библиотек, позволяющий восстанавливать цветное изображение по раздельно снятым цветным каналам
  \end{itemize}
}

\cventry{03/2010--$\infty$}{Разработчик}{C++/JavaScript/Google V8}{оpen-source разработки}{}{Разработка приложений и модулей для Node.js.\newline{}%
  Проекты:%
  \begin{itemize}%
    \item Разработка асинхронного модуля для обращения к MySQL
  \end{itemize}
}

\section{Образование}
\cventry{2005--2011}{МГУ им.~М.В.~Ломоносова, физический факультет}{Москва}{специалист}{средний бал 4.5}{Кафедра общей физики и волновых процессов.}

\section{Языки}
\cvlanguage{Русский}{Родной}{}
\cvlanguage{Английский}{Технический}{Относительно свободное чтение технической литературы в~области физики, математикик и компьютерной техники.
                                      При общении c авторами проектов, в~которых я участвовал на Github, также проблем с пониманием не возникало.}

\end{document}
